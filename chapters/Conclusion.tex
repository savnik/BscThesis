%!TEX root = ../Thesis.tex
\chapter{Conclusion}
The problem formulation was analysed resulting in a design requirement specification. The design requirement specification have been used to analytically design a prototype and discussing several methods. The prototype have been tested and the results clearly indicates the measuring devises works as intended with an acceptable precision.  Several elements are subject to further development.


\section{Further Work}
There is no doubt this work can be improved and develop further. 

\subsection*{Rearange load cells on UAV}
The load cells on the UAV can be rearranged to ensure the center of gravity and the cable connection coming closer to each other, and thus improves the UAV's stability.


\subsection*{Power system}
At the moment the power supply system are only able to deliver 400W insted of 500W over only 20m of cable. The limiting factor is in the converters/inverters system developed by Mikkel Wahlgreen. A way of solving this problem is to combine the power supply from the cable with a on board battery. When the UAV not are using all the power it can recharge the battery and when extra power is need is can be delivered in combination with the battery.

\subsection{RHD Link}

\subsection{Joystick}

\subsection{Network connection}

\subsection{Serial connection to PixHawk}