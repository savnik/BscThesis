%!TEX root = ../Thesis.tex
\chapter{Conclusion}
The thesis statement was split into components and analysed resulting in a design requirement specification. The design requirement specification have been used to analytically design a prototype and discussing several methods. For this project the simple solution was prototyped. The prototype have been tested and the results clearly indicates the measuring devises works as intended with an acceptable precision range. The data connection turned out to be stable under the testing conditions. Unfortunately problems with the power supply system obstructed testing the system with the UAV airborne.
Several elements are subject to further development.


\section{Further Work}
Every project can always be improved and further developed. To extend the work presented, following suggestions are obvious to pursue.

\subsection*{Rearrange load cells on UAV}
The load cells on the UAV can be rearranged to ensure the center of gravity and the cable connection coming closer to each other, and thus improves the UAV's stability.


\subsection*{Power system}
At the moment the power supply system is only able to deliver 400W instead of 500W over only 20m of cable. The limiting factor is in the converters/inverters system developed by Mikkel Wahlgreen. A way of solving this problem is to combine the power supply from the cable with a on board battery. When the UAV not are using all the power it can recharge the battery and when extra power is need is can be delivered in combination with the battery.

\subsection{RHD Link}
The link between the Ground Control Station and Flight Control Station is using the RHD server connection. However the RHD software is not intended to be used at two different computers and then merge the variable database. Many programming issues turned out on the way and the time was not to finish implementing this plugin.  

\subsection{Joystick}
The Joystick control is a nice feature while developing and testing the setup. Unfortunately the Joystick linux driver Xpad is not compiled with the distributed debian build for Beaglebone. This issue is discussed in the General Notes for Beaglebone setup in the Appendix. 

\subsection{Network connection}
The network connection has to be tested under the worst-case conditions, in order to safely determine the stability of the connection when the voltage drop.

\subsection{Serial connection to PixHawk}
This project relies on the Beaglebone at the UAV can communicate with the PixHawk, in order to feed the position controller with the measured positions references. Hence the USB connection on the Beaglebone is already used to the Phidget bridge, a serial communication to the PixHawk is preferred.

